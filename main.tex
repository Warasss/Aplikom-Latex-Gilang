\documentclass{report}

\usepackage[utf8]{inputenc}
\usepackage{eumat}
\usepackage[Conny]{fncychap}
\usepackage[bahasa]{babel}

% Rename Contents
\addto\captionsenglish{\renewcommand{\contentsname}{\vspace{-0.5cm} \textbf{Daftar Isi} \vspace{-2cm}}}

\begin{document}

% Cover Page
\begin{titlepage}
    \begin{center}
        \vspace*{1cm}

        \Huge
        \textbf{Tugas Akhir}
        
        \vspace{0.5cm}
        
        \LARGE
        Tugas Akhir Aplikasi Komputer  
        
        \vspace{1cm}
        
        \includegraphics[width=0.5\textwidth]{Logo UNY.png}

        \vspace{1cm}
        
        \textbf{Gilang Rizkiawan}\\
        22305144017\\
        Matematika E 2022
        
        \vspace{2cm}
        
        \Large
        \textbf{PRODI MATEMATIKA}\\
        \textbf{JURUSAN PENDIDIKAN MATEMATIKA}\\
        \textbf{FAKULTAS MATEMATIKA DAN ILMU PENGETAHUAN ALAM}
        \textbf{UNIVERSITAS NEGERI YOGYAKARTA}\\
        \textbf{2023}
        
    \end{center}
\end{titlepage}

\newpage
\tableofcontents

\chapter{KB Pekan 2 (Belajar Menggunakan Software EMT)}
\input{Pekan 2/EMT1Firstep_Gilang Rizkiawan}

\newpage
\chapter{KB Pekan 3-4: Menggunakan EMT untuk menyelesaikan masalah-masalah Aljabar}
\input{Pekan 3-4/EMT2Aljabar_Gilang Rizkiawan}

\newpage
\chapter{KB Pekan 5-6: Menggunakan EMT untuk mengambar grafik 2 dimensi (2D)}
\input{Pekan 5-6/EMT3Plot2D_Gilang Rizkiawan}

\newpage
\chapter{KB Pekan 7-8: Menggunakan EMT untuk mengambar grafik 3 dimensi (3D)}
\input{Pekan 7-8/EMT3Plot3D_Gilang Rizkiawan}

\newpage
\chapter{KB Pekan 9-10: Menggunakan EMT untuk kalkulus}
\input{Pekan 9-10/EMT5Kalkulus_Gilang Rizkiawan}

\newpage
\chapter{KB Pekan 11-12: Menggunakan EMT untuk Geometri}
\input{Pekan 11-12/EMT4Geometry_21305141002}

\newpage
\chapter{KB Pekan 13-14; Menggunakan EMT untuk Statistika}
\input{Pekan 12-13/EMT7Statistika_Gilang Rizkiawan}

\end{document}

\end{document}